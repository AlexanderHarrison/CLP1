%
% Copyright 2018 Joel Feldman, Andrew Rechnitzer and Elyse Yeager.
% This work is licensed under a Creative Commons Attribution-NonCommercial-ShareAlike 4.0 International License.
% https://creativecommons.org/licenses/by-nc-sa/4.0/
%
\questionheader{ex:s3.3.1}

%%%%%%%%%%%%%%%%%%
\subsection*{\Conceptual}
%%%%%%%%%%%%%%%%%%


\begin{question}
Which of the following is a differential equation for an unknown function $y$ of $x$?
\[\mbox{(a) } y=\diff{y}{x}
\qquad\mbox{(b) } \diff{y}{x}=3\left[y-5\right]
\qquad\mbox{(c) } y=3\left[y-\diff{x}{x}\right]
\qquad\mbox{(d) } e^x=e^y+1
\qquad\mbox{(e) } y=10e^x\]
\end{question}
\begin{hint}
Review the definition of a differential equation at the beginning of this section.
\end{hint}
\begin{answer} (a), (b)
\end{answer}
\begin{solution}
In the beginning of this section, the text says ``A differential equation is an equation for an unknown function that involves the derivative
of the unknown function." Our unknown function is $y$, so a differential equation is an equation that relates $y$ and $\ds\diff{y}{x}$. This applies to (a) and (b), but not (c), (d), or (e).

Note that $\ds\diff{x}{x}=1$: this is the derivative of $x$ with respect to $x$.
\end{solution}



\begin{Mquestion}
Which of the following functions $Q(t)$ satisfy the differential equation $Q(t)=5\ds\diff{Q}{t}$?
\[\mbox{(a) } Q(t)=0\qquad
\mbox{(b) } Q(t)=5e^t\qquad
\mbox{(c) } Q(t)=e^{5t}\qquad
\mbox{(d) } Q(t)=e^{t/5}\qquad
\mbox{(e) } Q(t)=e^{t/5}+1\]
\end{Mquestion}
\begin{hint}
You can test whether a given function solves a differential equation
      by substituting the function into the equation.
\end{hint}
\begin{answer} (a), (d)
\end{answer}
\begin{solution}
Theorem~\ref*{thm:growthDEsoln} tells us that a function is a solution to the differential equation $\ds\diff{Q}{t}=kQ(t)$ if and only if the function has the form $Q(t)=Ce^{kt}$ for some constant $C$. In our case, we want $Q(t)=5\ds\diff{Q}{t}$, so $\ds\diff{Q}{t}=\dfrac{1}{5}Q(t)$. So, the theorem tells us that the solutions are the functions of the form $Q(t)=Ce^{t/5}$. This applies to (a) (with $C=0$) and (d) (with $C=1$), but none of the other functions.

We don't actually need a theorem to answer this question, though: we can just test every option.
\begin{itemize}
\item[(a)] $\ds\diff{Q}{t}=0$, so $Q(t)=0=5\cdot 0 = 5\ds\diff{Q}{t}$, so (a) is a solution.
\item[(b)] $\ds\diff{Q}{t}=5e^t=Q(t)$, so $Q(t)=\ds\diff{Q}{t} \neq 5 \ds\diff{Q}{t}$, so (b) is not a solution.
\item[(c)] $\ds\diff{Q}{t}=5e^{5t}=5Q(t)$, so $Q(t)=\dfrac{1}{5}\ds\diff{Q}{t}\neq5\ds\diff{Q}{t}$, so (c) is not a solution.
\item[(d)] $\ds\diff{Q}{t}=\frac{1}{5}e^{t/5}=\frac{1}{5}Q(t)$, so
$Q(t)=5\ds\diff{Q}{t}$, so (d) is a solution.
\item[(e)] $\ds\diff{Q}{t}=\frac{1}{5}e^{t/5}=\frac{1}{5}\left(Q(t)-1\right)$, so
$Q(t)=5\ds\diff{Q}{t}+1$, so (e) is not a solution.
\end{itemize}
\end{solution}


\begin{Mquestion}
Suppose a sample starts out with $C$ grams of a radioactive isotope, and the amount of
the radioactive isotope left in the sample at time $t$ is given by
\[Q(t)=Ce^{-kt}\]
for some positive constant $k$. When will $Q(t)=0$?
\end{Mquestion}
\begin{hint}
Solve $0=Ce^{-kt}$ for $t$.
\end{hint}
\begin{answer}
If $C=0$, then there was none to start out with, and $Q(t)=0$ for all values of $t$.\\
If $C \neq 0$, then $Q(t)$ will never be 0 (but as $t$ gets bigger and bigger, $Q(t)$ gets closer and closer to 0).
\end{answer}
\begin{solution}
What we're asked to find is when
\begin{align*}
Q(t)&=0
\intertext{That is,}
Ce^{-kt}&=0
\end{align*}
If $C=0$, then this is the case for all $t$. There was no isotope to begin with, and there will continue not being any undecayed isotope forever.

If $C>0$, then since $e^{-kt}>0$, also $Q(t)>0$: so $Q(t)$ is never 0 for any value of $t$.
(But as $t$ gets bigger and bigger, $Q(t)$ gets closer and closer to 0.)

Remark: The last result is somewhat disturbing: surely at some point the last atom has decayed. The differential equation we use is a model that assumes $Q$ runs continuously. This is a good approximation only
      when there is a very large number of atoms. In practice, that is
      almost always the case.
\end{solution}


%%%%%%%%%%%%%%%%%%
\subsection*{\Procedural}
%%%%%%%%%%%%%%%%%%



\begin{question}[2015Q]
Consider a function of the form $f(x) = A e^{kx}$ where $A$ and $k$ are constants.
If $f(0)=5$ and $f(7)=\pi$, find the constants $A$ and $k$.
\end{question}
\begin{hint} No calculus here--just a review of the algebra of exponentials.
\end{hint}
\begin{answer}  $A=5$, $k = \dfrac{1}{7} \cdot \log\left(\pi/5\right)$
\end{answer}
\begin{solution}
The two pieces of information give us
\begin{align*}
  f(0) &= A = 5 & f(7) &= A e^{7k}=\pi
\end{align*}
Thus we know that $A=5$ and so $\pi = f(7) = 5e^{7k}$. Hence
\begin{align*}
  e^{7k} &= \frac{\pi}{5} \\
  7k &= \log(\pi/5) \\
  k &= \frac{1}{7} \cdot \log(\pi/5).
\end{align*}
where we use $\log$ to mean natural logarithm, $\log_e$.
\end{solution}



\begin{question}[2007H]
 Find the function $y(t)$ if $\ds\diff{y}{t} +3y = 0$, $y(1) = 2$.
\end{question}
\begin{hint}
Use Theorem~\ref*{thm:growthDEsoln}.
\end{hint}
\begin{answer}
$y(t)=2e^{-3(t-1)}$, or equivalently, $y(t)=2e^3e^{-3t}$
\end{answer}
\begin{solution}
In Theorem~\ref*{thm:growthDEsoln}, we saw that if $y$ is a function of $t$, and
$\ds\diff{y}{t}=-ky$, then
$y=Ce^{-kt}$ for some constant $C$.

Our equation $y$ satisfies $\ds\diff{y}{t}=-3y$, so the theorem tells us $y=Ce^{-3t}$ for some constant $C$.

We are also told that $y(1)=2$. So, $2=Ce^{-3 \times 1}$ tells us $C=2e^3$. Then:
\[y=2e^{3}\cdot e^{-3t}=2e^{-3(t-1)}.\]
\end{solution}



\begin{question}
A sample of bone belongs to an animal that died 10,000 years ago. If the bone contained
5 $\mu$g of Carbon-14 when the animal died, how much Carbon-14 do you expect it to have now?
\end{question}
\begin{hint}
From the text, we see the half-life of Carbon-14 is 5730 years. A microgram ($\mu$g) is one-millionth of a gram, but you don't need to know that to solve this problem.
\end{hint}
\begin{answer}
$5\cdot 2^{-\tfrac{10000}{5730}}\approx 1.5~\mu g$
\end{answer}
\begin{solution}
The amount of Carbon-14 in the sample $t$ years after the animal died will be
\[Q(t)=5e^{-kt}\]
for some constant $k$ (where 5 is the amount of Carbon-14 in the sample at time $t=0$). So, the answer we're looking for is $Q(10000)$. We need to replace $k$ with an actual number to evaluate $Q(10000)$, and the key to doing this is the half-life. The text tells us that the half-life of Carbon-14 is 5730 years, so we know:
\begin{align*}
Q(5730)&=\frac{5}{2}\\
5e^{-k\cdot5730}&=\frac{5}{2}\\
\left(e^{-k}\right)^{5730}&=\frac{1}{2}\\
e^{-k}&=\sqrt[5730]{\frac{1}{2}}=2^{-\tfrac{1}{5730}}
\intertext{So:}
Q(t)&=5\left(e^{-k}\right)^t\\
&=5\cdot2^{-\tfrac{t}{5730}}
\intertext{Now, we can evaluate:}
Q(10000)&=5\cdot 2^{-\tfrac{10000}{5730}}\approx 1.5~\mu g
\end{align*}
Remark: after $2(5730)=11,460$ years, the sample will have been sitting for two half-lives, so its remaining Carbon-14 will be a quarter of its original amount, or $1.25$ $\mu$g. It makes sense that at 10,000 years, the sample will contain slightly more Carbon-14 than at 11,460 years. Indeed, 1.5 is slightly larger than 1.25, so our answer seems plausible.

It's a good habit to look for ways to quickly check whether your answer seems plausible, since a small algebra error can easily turn into a big error in your solution.
\end{solution}






\begin{Mquestion}
A sample containing one gram of Radium-226 was stored in a lab 100 years ago; now the sample only contains 0.9576 grams of Radium-226. What is the half-life of Radium-226?
\end{Mquestion}
\begin{hint}
The quantity of Radium-226 in the sample at time $t$ will be $Q(t)=Ce^{-kt}$ for some positive constants $C$ and $k$. You can use the given information to find $C$ and $e^{-k}$.

In the following work, remember we use $\log$ to mean natural logarithm, $\log_e$.
\end{hint}
\begin{answer}
Radium-226 has a half life of about 1600 years.
\end{answer}
\begin{solution}
Let 100 years ago be the time $t=0$. Then if $Q(t)$ is the amount of Radium-226 in the sample, $Q(0)=1$, and
\begin{align*}
Q(t)&=e^{-kt}
\intertext{for some positive constant $k$. When $t=100$, the amount of Radium-226 left is 0.9576 grams, so}
0.9576=Q(100)&=e^{-k\cdot 100}=\left(e^{-k}\right)^{100}\\
e^{-k}&=0.9576^{\tfrac{1}{100}}\intertext{This tells us}
Q(t)&=0.9576^{\tfrac{t}{100}}
\intertext{So, if half the original amount of Radium-226 is left,}
\frac{1}{2}&=0.9576^{\tfrac{t}{100}}\\
\log\left(\frac{1}{2}\right)&=\log\left(0.9576^{\tfrac{t}{100}}\right)\\
-\log 2&=\frac{t}{100}\log(0.9576)\\
t&=-100\frac{\log 2}{\log 0.9576}\approx 1600
\end{align*}
So, the half life of Radium-226 is about 1600 years.
\end{solution}


\begin{question}[2006H]
The mass of a sample of Polonium--210, initially 6 grams,
decreases at a rate proportional to the mass. After one year, 1 gram remains.
What is the half--life (the time it takes for the sample to decay to half
its original mass)?
\end{question}
\begin{hint}
The fact that the mass of the sample decreases at a rate proportional to its mass tells us that, if $Q(t)$ is the mass of Polonium-201, the following differential equation holds:
\[\diff{Q}{t}=-kQ(t)\]
where $k$ is some positive constant. Compare this to Theorem~\ref*{thm:growthDEsoln}.
\end{hint}
\begin{answer} $\dfrac{\log 2}{\log 6}=\log_6(2)$ years, which is about 139 days
\end{answer}
\begin{solution}
Let $Q(t)$ denote the mass at time $t$.
Then $\ds\diff{Q}{t}$ is the rate at which the mass is changing. Since the rate the mass is decreasing is proportional to the mass remaining, we know $\ds\diff{Q}{t}=-kQ(t)$, where $k$ is a positive constant. (Remark: since $Q$ is decreasing, $\ds\diff{Q}{t}$ is negative. Since we cannot have a negative mass, if we choose $k$ to be positive, then $k$ and $Q$ are both positive--this is why we added the negative sign.)

The information given in the question is:
$$
Q(0)=6\qquad
\diff{Q}{t}=-kQ(t)\qquad
Q(1)=1
$$
for some constant $k>0$.
By Theorem~\ref*{thm:growthDEsoln}, we know \[Q(t)=Ce^{-kt}\] for some constant $C$.
Since $Q(0)=Ce^{0}=C$, the given information tells us $6=C$. (This is the initial mass of our sample.) So, $Q(t)=6e^{-kt}$. To get the full picture of the behaviour of $Q$, we should find $k$. We do this using the given information $Q(1)=1$:
\begin{align*}
1&=Q(1)=6e^{-k(1)}\\
6^{-1}=\frac{1}{6}&=e^{-k}\\
%6&= e^k\\
%\log 6 &=k
\intertext{So, all together,}
Q(t)&=6\left(e^{-k}\right)^t=6\cdot \left(6^{-1}\right)^t=6^{1-t}
\end{align*}

The question asks us to determine the time $t_{h}$ which obeys
$Q(t_{h})=\dfrac{6}{2}=3$. Now that we know the equation for $Q(t)$, we simply solve:
\begin{align*}
Q(t)&=6^{1-t}\\
3=Q(t_h)&=6^{1-t_{h}}\\
\log 3 &=\log\left(6^{1-t_h}\right)=(1-t_h)\log 6\\
\frac{\log 3}{\log 6}&=1-t_h\\
t_h&=1-\frac{\log 3}{\log 6}=\frac{\log 6-\log 3}{\log 6}=\frac{\log 2}{\log 6}
\end{align*}
The half-life of Polonium-210 is $\dfrac{\log 2}{\log 6}$ years, or about 141 days.

Remark:  The actual half-life of Polonium-210 is closer to 138 days. The numbers in the question are made to work out nicely, at the expense of some accuracy.
\end{solution}




\begin{question}
Radium-221 has a half-life of 30 seconds. How long does it take for only 0.01\% of an original sample to be left?
\end{question}
\begin{hint}
The amount of Radium-221 in a sample at time $t$ will be $Q(t)=Ce^{-kt}$ for some positive constants $C$ and $k$. You can leave $C$ as a variable--it's the original amount in the sample, which isn't specified. What you want to find is the value of $t$ such that\\ $Q(t)=0.0001Q(0)=0.0001C$.
\end{hint}
\begin{answer}
$120\cdot\dfrac{\log 10}{\log 2}$ seconds, or about six and a half minutes.
\end{answer}
\begin{solution}
The amount of Radium-221 in a sample will be
\begin{align*}
Q(t)&=Ce^{-kt}
\intertext{where $C$ is the amount in the sample at time $t=0$, and $k$ is some positive constant. We know the half-life of the isotope, so we can find $e^{-k}$:}
\frac{C}{2}=Q(30)&=Ce^{-k\cdot 30}\\
\frac{1}{2}&=\left(e^{-k}\right)^{30}\\
2^{-\tfrac{1}{30}}&=e^{-k}
\intertext{So,}
Q(t)&=C\left(e^{-k}\right)^t=C\cdot2^{-\tfrac{t}{30}}
\intertext{When only 0.01\% of the original sample is left, $Q(t)=0.0001C$:}
0.0001C=Q(t)&=C\cdot 2^{-\tfrac{t}{30}}\\
0.0001&=2^{-\tfrac{t}{30}}\\
\log(0.0001)&=\log\left(2^{-\tfrac{t}{30}}\right)\\
\log\left(10^{-4}\right)&=-\frac{t}{30}\log2\\
-4\log 10&=-\frac{t}{30}\log2\\
t&=120\cdot\frac{\log10}{\log2}\approx 398.6
\end{align*}
It takes about 398.6 seconds (that is, roughly 6 and a half minutes) for all but $0.01$\% of the sample to decay.

Remark: we can do another reality check here. The half-life is 30 seconds. 6 and a half minutes represents 13 half-lives. So, the sample is halved 13 times: $\left(\tfrac{1}{2}\right)^{13}\approx 0.00012=0.012\%$. So these 13 half-lives should reduce the sample to about 0.01\% of its original amount, as desired.
\end{solution}


%%%%%%%%%%%%%%%%%%
\subsection*{\Application}
%%%%%%%%%%%%%%%%%%

\begin{Mquestion}
Polonium-210 has a half life of 138 days. What percentage of a sample of Polonium-210 decays in a day?
\end{Mquestion}
\begin{hint}
You don't need to know the original amount of Polonium-210 in order to answer this question: you can leave it as some constant $C$, or you can call it 100\%.
\end{hint}
\begin{answer}
About $0.5$\% of the sample decays in a day. The exact amount is
$\left[100\left(1-2^{-\tfrac{1}{138}}\right)\right]$\%.
\end{answer}
\begin{solution}
We know that the amount of Polonium-210 in a sample after $t$ days is given by
\begin{align*}
Q(t)&=Ce^{-kt}
\intertext{where $C$ is the original amount of the sample, and $k$ is some positive constant.}
\intertext{The question asks us what percentage of the sample decays in a day. Since $t$ is measured in days, the amount that decays in a day is $Q(t)-Q(t+1)$. The percentage of $Q(t)$ that this represents is $100\dfrac{Q(t)-Q(t+1)}{Q(t)}$. (For example, if there were two grams at time $t$, and one gram at time $t+1$, then $100\dfrac{2-1}{1}=50$: 50\% of the sample decayed in a day.)}
\intertext{In order to simplify, we should figure out a better expression for $Q(t)$. As usual, we make use of the half-life.}
Q(138)&=\frac{C}{2}\\
Ce^{-k\cdot138}&=\frac{C}{2}\\
\left(e^{-k}\right)^{138}&=\frac{1}{2}=2^{-1}\\
e^{-k}&=2^{-\tfrac{1}{138}}
\intertext{Now, we have a better formula for $Q(t)$:}
Q(t)&=C\left(e^{-k}\right)^t\\
Q(t)&=C\cdot2^{-\tfrac{t}{138}}
\intertext{Finally, we can evaluate what percentage of the sample decays in a day.}
100\frac{Q(t)-Q(t+1)}{Q(t)}&=100\frac{C\cdot2^{-\tfrac{t}{138}}-C\cdot2^{-\tfrac{t+1}{138}}}{C\cdot2^{-\tfrac{t}{138}}}\left(\frac{\frac{1}{C}}{\frac{1}{C}}\right)\\
&=100\frac{2^{-\tfrac{t}{138}}-2^{-\tfrac{t+1}{138}}}{2^{-\tfrac{t}{138}}}\\
&=100\left(2^{-\tfrac{t}{138}}-2^{-\tfrac{t+1}{138}}\right)2^{\tfrac{t}{138}}\\
&=100\left(1-2^{-\tfrac{1}{138}}\right)\approx 0.5
\end{align*}
About 0.5\% of the sample decays in a day.

Remark: when we say that half a percent of the sample decays in a day, we don't mean half a percent of the \emph{original} sample. If a day starts out with, say, 1 microgram, then what decays in the next 24 hours is about half a percent of that 1 microgram, regardless of what the ``original" sample (at some time $t=0$) held.

In particular, since the sample is getting smaller and smaller, that half of a percent that decays every day represents fewer and fewer actual atoms decaying. That's why we can't say that half of the sample (50\%) will decay after about 100 days, even though 0.5\% decays every day and $100\times 0.5 = 50$.
\end{solution}




\begin{question}
A sample of ore is found to contain $7.2 \pm 0.3~\mu$g of Uranium-232, the half-life of which is between 68.8 and 70 years. How much Uranium-232 will remain undecayed in the sample in 10 years?
\end{question}
\begin{hint}
Try to find the most possible and least possible remaining Uranium-232, given the bounds in the problem.
\end{hint}
\begin{answer}
After ten years, the sample contains between 6.2 and 6.8 $\mu$g of Uranium-232.
\end{answer}
\begin{solution}
The amount of Uranium-232 in the sample of ore at time $t$ will be
\begin{align*}
Q(t)&=Q(0)e^{-kt}
\intertext{where $6.9 \leq Q(0) \leq 7.5$. We don't exactly know $Q(0)$, and we don't exactly know the half-life, so we also won't exactly know $Q(10)$: we can only say that is it between two numbers. Our strategy is to find the highest and lowest possible values of $Q(10)$, given the information in the problem.}
\intertext{In order for the most possible Uranium-232 to be in the sample after 10 years, we should start with the most and have the longest half-life (since this represents the slowest decay). So, we take $Q(0)=7.5$ and $Q(70)=\frac{1}{2}(7.5)$.}
Q(t)&=7.5e^{-kt}\\
\frac{1}{2}(7.5)=Q(70)&=7.5e^{-k(70)}\\
\frac{1}{2}&=\left(e^{-k}\right)^{70}\\
2^{-\tfrac{1}{70}}&=e^{-k}
\intertext{So, in this secenario,}
Q(t)&=7.5\cdot 2^{-\tfrac{t}{70}}
\intertext{After ten years,}
Q(10)&=7.5\cdot 2^{-\tfrac{10}{70}}\approx 6.79
\intertext{So after ten years, the sample contains \emph{at most} 6.8 $\mu$g.}
\intertext{Now, let's think about the least possible amount of Uranium-232 that could be left after 10 years. We should start with as little as possible, so take $Q(0)=6.9$, and the sample should decay quickly, so take the half-life to be 68.8 years.}
Q(t)&=6.9e^{-kt}\\
\frac{1}{2}6.9=Q(68.8)&=6.9e^{-k(68.8)}\\
\frac{1}{2}&=\left(e^{-k}\right)^{68.8}\\
2^{-\tfrac{1}{68.8}}&=e^{-k}
\intertext{In this scenario,}
Q(t)&=6.9\cdot 2^{-\tfrac{t}{68.8}}
\intertext{After ten years,}
Q(10)&=6.9\cdot 2^{-\tfrac{10}{68.8}}\approx 6.24
\intertext{So after ten years, the sample contains \emph{at least} 6.2 $\mu$g.}
\end{align*}
After ten years, the sample contains between 6.2 and 6.8 $\mu$g of Uranium-232.
\end{solution}
