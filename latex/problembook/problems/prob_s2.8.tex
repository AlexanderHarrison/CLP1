%
% Copyright 2018 Joel Feldman, Andrew Rechnitzer and Elyse Yeager.
% This work is licensed under a Creative Commons Attribution-NonCommercial-ShareAlike 4.0 International License.
% https://creativecommons.org/licenses/by-nc-sa/4.0/
%
\questionheader{ex:s2.8}

%%%%%%%%%%%%%%%%%%
\subsection*{\Conceptual}
%%%%%%%%%%%%%%%%%%

\begin{Mquestion}
Graph sine and cosine on the same axes, from $x=-2\pi$ to $x=2\pi$. Mark the points where $\sin x$ has a horizontal tangent. What do these points correspond to, on the graph of cosine?
\end{Mquestion}
\begin{hint}
A horizontal tangent line is where the graph appears to ``level off."
\end{hint}
\begin{answer}
\begin{center}\begin{tikzpicture}
\YEaxis{7}{1.2}
\draw (3.1,.2)--(3.1,-.2) node[below]{$\pi$};
\draw (-3.1,.2)--(-3.1,-.2) node[below]{$-\pi$};
\draw[thick, blue] plot[domain=-6.28:0](\x,{sin \x r});
\draw[thick, blue] plot[domain=-6.28:0](\x+6.28,{sin \x r}) node[above]{$y=\sin x$};
\foreach \x in {-4.71,1.57}
	{\draw[blue] ( \x,1) node[vertex]{};
	\draw[blue, dashed] (\x-.75,1)--(\x+.75,1);}
\foreach \x in {4.71,-1.57}
	{\draw[blue] ( \x,-1) node[vertex]{};
	\draw[blue, dashed] (\x-.75,-1)--(\x+.75,-1);}
\draw[thick, red] plot[domain=-6.28:0](\x,{cos \x r});
\draw[thick, red] plot[domain=-6.28:0](\x+6.28,{cos \x r}) node[above]{$y=\cos x$};
\foreach \x in {-4.71,4.71,1.57,-1.57}
	{\draw[red] (\x,0) node[vertex]{};}
\end{tikzpicture}\end{center}
The graph $f(x)=\sin x$ has horizontal tangent lines precisely at those points where $\cos x=0$.
\end{answer}
\begin{solution}
\begin{center}\begin{tikzpicture}
\YEaxis{7}{1.2}
\draw (3.1,.2)--(3.1,-.2) node[below]{$\pi$};
\draw (-3.1,.2)--(-3.1,-.2) node[below]{$-\pi$};
\draw[thick, blue] plot[domain=-6.28:0](\x,{sin \x r});
\draw[thick, blue] plot[domain=-6.28:0](\x+6.28,{sin \x r}) node[above]{$y=\sin x$};
\foreach \x in {-4.71,1.57}
	{\draw[blue] ( \x,1) node[vertex]{};
	\draw[blue, dashed] (\x-.75,1)--(\x+.75,1);}
\foreach \x in {4.71,-1.57}
	{\draw[blue] ( \x,-1) node[vertex]{};
	\draw[blue, dashed] (\x-.75,-1)--(\x+.75,-1);}
\draw[thick, red] plot[domain=-6.28:0](\x,{cos \x r});
\draw[thick, red] plot[domain=-6.28:0](\x+6.28,{cos \x r}) node[above]{$y=\cos x$};
\foreach \x in {-4.71,4.71,1.57,-1.57}
	{\draw[red] (\x,0) node[vertex]{};}
\end{tikzpicture}\end{center}
The graph $f(x)=\sin x$ has horizontal tangent lines precisely at those points where $\cos x=0$. This must be true, since $\diff{}{x}\{\sin x\}=\cos x$: where the derivative of sine is zero, cosine itself is zero.
\end{solution}

\begin{Mquestion}
Graph sine and cosine on the same axes, from $x=-2\pi$ to $x=2\pi$. Mark the points where $\sin x$ has a tangent line of maximum (positive) slope. What do these points correspond to, on the graph of cosine?
\end{Mquestion}
\begin{hint}
You are going to mark there points on the sine graph where the graph is the steepest, going up.
\end{hint}
\begin{answer}
\begin{center}\begin{tikzpicture}
\YEaxis{7}{1.2}
\draw (3.1,.2)--(3.1,-.2) node[below]{$\pi$};
\draw (-3.1,.2)--(-3.1,-.2) node[below]{$-\pi$};
\draw[thick, blue] plot[domain=-6.28:0](\x,{sin \x r});
\draw[thick, blue] plot[domain=-6.28:0](\x+6.28,{sin \x r}) node[above]{$y=\sin x$};
\foreach \x in {-6.28,0,6.28}
	{\draw[blue] ( \x,0) node[vertex]{};
	\draw[blue, dashed] (\x-1,-1)--(\x+1,1);}
\draw[thick, red] plot[domain=-6.28:0](\x,{cos \x r});
\draw[thick, red] plot[domain=-6.28:0](\x+6.28,{cos \x r}) node[above]{$y=\cos x$};
\foreach \x in {-6.28,0,6.28}
	{\draw[red] (\x,1) node[vertex]{};}
\end{tikzpicture}\end{center}
The graph $f(x)=\sin x$ has maximum slope at those points where $\cos x$ has a maximum. That is, where $\cos x = 1$.
\end{answer}
\begin{solution}
\begin{center}\begin{tikzpicture}
\YEaxis{7}{1.2}
\draw (3.1,.2)--(3.1,-.2) node[below]{$\pi$};
\draw (-3.1,.2)--(-3.1,-.2) node[below]{$-\pi$};
\draw[thick, blue] plot[domain=-6.28:0](\x,{sin \x r});
\draw[thick, blue] plot[domain=-6.28:0](\x+6.28,{sin \x r}) node[above]{$y=\sin x$};
\foreach \x in {-6.28,0,6.28}
	{\draw[blue] ( \x,0) node[vertex]{};
	\draw[blue, dashed] (\x-1,-1)--(\x+1,1);}
\draw[thick, red] plot[domain=-6.28:0](\x,{cos \x r});
\draw[thick, red] plot[domain=-6.28:0](\x+6.28,{cos \x r}) node[above]{$y=\cos x$};
\foreach \x in {-6.28,0,6.28}
	{\draw[red] (\x,1) node[vertex]{};}
\end{tikzpicture}\end{center}
The graph $f(x)=\sin x$ has maximum slope at those points where $\cos x$ has a maximum. This makes sense, because $f'(x)=\cos x$: the maximum values of the slope of sine correspond to the maximum values of cosine.
\end{solution}

%%%%%%%%%%%%%%%%%%
\subsection*{\Procedural}
%%%%%%%%%%%%%%%%%%


\begin{question}
Differentiate $f(x)=\sin x + \cos x +\tan x$.
\end{question}
\begin{hint}
You need to memorize the derivatives of sine, cosine, and tangent.
\end{hint}
\begin{answer}
$f'(x)=\cos x - \sin x + \sec^2 x$
\end{answer}
\begin{solution}
You should memorize the derivatives of sine, cosine, and tangent.\\
$f'(x)=\cos x - \sin x + \sec^2 x$
\end{solution}


\begin{Mquestion}
For which values of $x$ does the function $f(x)=\sin x + \cos x$ have a horizontal tangent?
\end{Mquestion}
\begin{hint}
There are infinitely many values. You need to describe them all.
\end{hint}
\begin{answer}
$x=\frac{\pi}{4}+\pi n$, for any integer $n$.
\end{answer}
\begin{solution}
$f'(x)=\cos x - \sin x$, so $f'(x)=0$ precisely when $\sin x = \cos x$. This happens at $\pi/4$, but it also happens at $5\pi/4$. By looking at the unit circle, it is clear that $\sin x = \cos x$ whenever $x = \frac{\pi}{4}+\pi n$ for some integer $n$.
\begin{center}
\begin{tikzpicture}
\YEaxis{2.75}{2.75}
\draw[thick] node[shape=circle, minimum size=4cm, draw, inner sep=0]{};
\draw[dashed] (-2.75,-2.75)--(2.75,2.75);
\draw (1.41,1.41) node[vertex, label=right:{$\sin (\pi/4) = \cos (\pi/4)$}]{};
\draw (-1.41,-1.41) node[vertex, label=left:{$\sin (5\pi/4) = \cos (5\pi/4)$}]{};
\end{tikzpicture}
\end{center}
\end{solution}


\begin{Mquestion}
Differentiate $f(x)=\sin^2 x + \cos^2 x$.
\end{Mquestion}
\begin{hint}
Simplify first.
\end{hint}
\begin{answer}
0
\end{answer}
\begin{solution}
\begin{itemize}
\item Solution 1: $f(x)=\sin^2x+\cos^2x=1$, so $f'(x)=\diff{}{x}\{1\}=0$.
\item Solution 2: Using the formula for the derivative of a squared function,
\begin{align*}
f'(x)=2\sin x \cos x + 2\cos x(- \sin x)=2\sin x \cos x - 2 \sin x \cos x =0.
\end{align*}
\end{itemize}
\end{solution}


\begin{question}
Differentiate $f(x)=2\sin x \cos x$.
\end{question}
\begin{hint}
The identity won't help you.
\end{hint}
\begin{answer}
$f'(x)=2(\cos^2 x - \sin ^2 x)$
\end{answer}
\begin{solution}
It is true that $2\sin x \cos x = \sin (2x)$, but we don't know the derivative of $\sin(2x)$. So, we use the product rule: \[f'(x)=2\cos x \cos x+2\sin x (-\sin x)=2(\cos^2 x - \sin ^2 x).\]
\end{solution}


\begin{question}
Differentiate $f(x)=e^x\cot x$.
\end{question}
\begin{answer}
$f'(x)=e^x(\cot x - \csc^2 x)$
\end{answer}
\begin{solution}
\begin{itemize}
\item Solution 1:
 using the product rule,
\[f'(x)=e^x\cot x + e^x(-\csc^2 x)=e^x(\cot x - \csc^2 x).\]
\item Solution 2:
using the formula from Question~\ref{s2.7expprod}, Section~\ref*{sec exp func},
\[f'(x)=e^x(\cot x - \csc^2 x).\]
\end{itemize}
\end{solution}


\begin{Mquestion}
Differentiate $f(x) = \dfrac{2\sin x + 3 \tan x}{\cos x + \tan x}$
\end{Mquestion}
\begin{hint}
Quotient rule
\end{hint}
\begin{answer}
$f'(x)=\dfrac{2+3 \sec x + 2 \sin x -2\tan x \sec x+3\sin x \tan x }{(\cos x + \tan x)^2}$
\end{answer}
\begin{solution}
We use the quotient rule.
\begin{align*}
f'(x)&=\dfrac{(\cos x + \tan x)(2\cos x + 3 \sec^2 x)-(2\sin x+3\tan x)(-\sin x + \sec^2 x)}{(\cos x + \tan x)^2}\\
&=\frac{2\cos^2x+3\cos x \sec^2 x + 2 \cos x \tan x + 3 \tan x \sec^2 x}{(\cos x + \tan x)^2}\\
&~~~~
+\frac{2\sin^2x-2\sin x \sec^2x+3\sin x \tan x -3\tan x \sec^2 x}{(\cos x + \tan x)^2}\\
&=\frac{2+3 \sec x + 2 \sin x -2\tan x \sec x+3\sin x \tan x }{(\cos x + \tan x)^2}
\end{align*}
\end{solution}

\begin{Mquestion}
Differentiate $f(x) = \dfrac{5\sec x+1}{e^x}$.
\end{Mquestion}
\begin{answer}
$f'(x)=\dfrac{5\sec x \tan x - 5 \sec x - 1}{e^x}$
\end{answer}
\begin{solution}
We use the quotient rule.
\begin{align*}
f'(x) &= \frac{e^x(5\sec x \tan x)-(5\sec x + 1) e^x}{(e^{x})^2}\\
&=\frac{5\sec x \tan x - 5 \sec x - 1}{e^x}
\end{align*}
\end{solution}

\begin{question}
Differentiate $f(x)=(e^x+\cot x)(5x^6-\csc x)$.
\end{question}
\begin{answer}
$f'(x)=(e^x+\cot x)(30x^5+\csc x \cot x)+(e^x-\csc^2x)(5x^6-\csc x)$
\end{answer}
\begin{solution}
We use the product rule:
\begin{align*}
f'(x)&=(e^x+\cot x)(30x^5+\csc x \cot x)+(e^x-\csc^2x)(5x^6-\csc x)
\end{align*}
\end{solution}

\begin{question}
Differentiate $f(\theta)=\sin\left(\frac{\pi}{2}-\theta \right)$.
\end{question}
\begin{hint}
Use an identity.
\end{hint}
\begin{answer}
$-\sin(\theta)$
\end{answer}
\begin{solution}
We don't know how to differentiate this function as it is written, but an identity helps us.
Since $\sin\left(\frac{\pi}{2}-\theta \right)=\cos \theta$, we see
$f'(\theta)=\diff{}{\theta}\{\cos \theta\}=-\sin(\theta)$.
\end{solution}

\begin{question}
Differentiate $f(x)=\sin(-x)+\cos(-x)$.
\end{question}
\begin{hint}
How can you move the negative signs to a location that you can more easily deal with?
\end{hint}
\begin{answer}
$f'(x)=-\cos x - \sin x$
\end{answer}
\begin{solution}
We know the derivative of $\sin x$, but not of $\sin(-x)$. So we re-write $f(x)$ using identities:
\begin{align*}
f(x)&=\sin(-x)+\cos(-x)\\
&=-\sin x + \cos x\\
f'(x)&=-\cos x - \sin x
\end{align*}
\end{solution}


\begin{question}
Differentiate $s(\theta)=\dfrac{\cos \theta + \sin \theta}{\cos \theta - \sin\theta}$.
\end{question}
\begin{hint} Apply the quotient rule.
\end{hint}
\begin{answer}
$\left(\dfrac{\cos\theta+\sin\theta}{\cos\theta-\sin\theta}\right)^2+1$
\end{answer}
\begin{solution}
We apply the quotient rule.
\begin{align*}
s'(\theta)&=\frac{(\cos \theta-\sin\theta)(-\sin\theta+\cos\theta)-(\cos\theta+\sin\theta)(-\sin\theta-\cos\theta)}{(\cos\theta-\sin\theta)^2}\\
&=\frac{(\cos \theta-\sin\theta)^2+(\cos\theta+\sin\theta)^2}{(\cos\theta-\sin\theta)^2}\\
&=1+\left(\frac{\cos\theta+\sin\theta}{\cos\theta-\sin\theta}\right)^2
\end{align*}
\end{solution}


\begin{question}[2007H]
Find the values of the constants $a$ and $b$ for which
\[
f(x) = \left\{
	\begin{array}{cc} \cos(x) & x\le 0\\
               ax + b   &  x> 0\end{array}
               \right.
\]
is differentiable everywhere.
\end{question}
\begin{hint}
The only spot to worry about is when $x=0$. For $f(x)$ to be differentiable, it must be continuous, so first find the value of $b$ that makes $f$ continuous at $x=0$. Then, find the value of $a$ that makes the derivatives from the left and right of $x=0$ equal to each other.
\end{hint}
\begin{answer}
$a=0$, $b=1$.
\end{answer}
\begin{solution}
In order for $f$ to be differentiable at $x=0$, it must also
be continuous at $x=0$. This forces
\[
\lim_{x\to 0^-}f(x) =  \lim_{x\to 0^+}f(x) =f(0)\qquad\text{or}\qquad
\lim_{x\to 0^-}\cos(x) =  \lim_{x\to 0^+}(ax+b) =1
\]
or $b=1$. In order for $f$ to be differentiable at $x=0$, we need the limit
\[
\lim_{h\to 0}\frac{f(0+h)-f(0)}{h}
\]
to exist. This is the case if and only if the two one--sided limits
\begin{align*}
\lim_{h\to 0^-}\frac{f(0+h)-f(0)}{h}
&=\lim_{h\to 0^-}\frac{\cos(h)-\cos(0)}{h}
\intertext{and}
\lim_{h\to 0^+}\frac{f(0+h)-f(0)}{h}
&=\lim_{h\to 0^+}\frac{(ah+b)-\cos(0)}{h}
=a\qquad\text{since $b=1$}
\intertext{exist and are equal.
Because $\cos(x)$ is differentiable at $x=0$ we have}
\lim_{h\to 0^-}\frac{\cos(h)-\cos(0)}{h} &= \diff{}{x}\cos(x)\bigg|_{x=0}
                                         = -\sin(x)\Big|_{x=0}=0
\end{align*}
So, we need $a=0$ and $b=1$.
\end{solution}




\begin{Mquestion}[2015Q]
Find the equation of the line tangent to the graph of $y=\cos(x)+2x$ at
$x=\dfrac{\pi}{2}$.
\end{Mquestion}
\begin{answer} $y -  \pi = 1\cdot (x-\pi/2)$
\end{answer}
\begin{solution}
We compute the derivative of $\cos(x)+2x$ as being $-\sin(x)+2$, which evaluated at
$x=\frac{\pi}{2}$ yields $-1+2=1$. Since we also compute
$\cos(\pi/2)+2(\pi/2)=0+\pi$, then the equation of the tangent line is
\begin{align*}
y -  \pi = 1\cdot (x-\pi/2).
\end{align*}
\end{solution}


%%%%%%%%%%%%%%%%%%
\subsection*{\Application}
%%%%%%%%%%%%%%%%%%

\begin{question}[2015Q]
 Evaluate $\displaystyle \lim_{x\to \pi/3}\left(
\dfrac{\cos(x)-1/2}{x-\pi/3}\right).$  Use any method.
\end{question}
\begin{hint} This looks like a derivative that you know how to compute.
\end{hint}
\begin{answer}  $-\sqrt{3}/2$
\end{answer}
\begin{solution}
This limit represents the derivative computed at $x=\pi/3$ of the function
$f(x)=\cos x$. Since the derivative of $f(x)$ is $-\sin x$, then its
value at $x=\pi/3$ is exactly $-\sqrt{3}/2$.
\end{solution}

\begin{question}
Show how you can use the quotient rule to find the derivative of tangent, if you already know the derivatives of sine and cosine.
\end{question}
\begin{hint}
$\tan \theta = \dfrac{\sin \theta}{\cos \theta}$
\end{hint}
\begin{answer}
\begin{align*}
\tan \theta &= \dfrac{\sin \theta}{\cos \theta}
\intertext{So, using the quotient rule,}
\diff{}{\theta}\{\tan \theta\}&=\frac{\cos\theta\cos\theta-\sin\theta(-\sin\theta)}{\cos^2\theta}
=\frac{\cos^2\theta+\sin^2\theta}{\cos^2\theta}\\
&=\left(\frac{1}{\cos \theta}\right)^2=\sec^2\theta
\end{align*}
\end{answer}
\begin{solution}
\begin{align*}
\tan \theta &= \dfrac{\sin \theta}{\cos \theta}
\intertext{So, using the quotient rule,}
\diff{}{\theta}\{\tan \theta\}&=\frac{\cos\theta\cos\theta-\sin\theta(-\sin\theta)}{\cos^2\theta}
=\frac{\cos^2\theta+\sin^2\theta}{\cos^2\theta}\\
&=\left(\frac{1}{\cos \theta}\right)^2=\sec^2\theta
\end{align*}
\end{solution}


\begin{question}[1997A]
 The derivative of the function
\[
f(x)=\left\{\begin{array}{ll}
ax+b& \mbox{for }x<0\\
            \frac{6\cos x}{2+\sin x+\cos x}& \mbox{for }x\ge 0
\end{array}\right.
\]
exists for all $x$. Determine the values of the constants $a$ and $b$.
\end{question}
\begin{hint} In order for a derivative to exist, the function must be continuous, and the derivative from the left must equal the derivative from the right.
\end{hint}
\begin{answer} $a=-\frac{2}{3}$, $b=2$
\end{answer}
\begin{solution}  In order for the function $f(x)$ to be continuous at $x=0$,
the left half formula $ax+b$ and the right half formula
$\dfrac{6\cos x}{2+\sin x+\cos x}$ must match up at $x=0$. This
forces
$$
a\times 0+b=\frac{6\cos 0}{2+\sin 0+\cos 0}=\frac{6}{3}
\implies \boxed{b=2}
$$

In order for the derivative $f'(x)$ to exist at $x=0$,
the limit $\ds\lim_{h \rightarrow 0} \dfrac{f(h)-f(0)}{h}$ must exist. In particular,
the limits $\ds\lim_{h \rightarrow 0^-} \dfrac{f(h)-f(0)}{h}$
and
$\ds\lim_{h \rightarrow 0^+} \dfrac{f(h)-f(0)}{h}$ must exist and be equal to each other.

When $h \to 0^-$, this means $h<0$, so $f(h)=ah+b=ah+2$. So:
\[\ds\lim_{h\to 0^-}\dfrac{f(h)-f(0)}{h}=\ds\lim_{h\to 0^-}\dfrac{(ah+2)-2}{h}=\left.\ds\diff{}{x}\left\{ax+2\right\}\right|_{x=0}=a.\]

Similarly, when $h \to 0^+$, then $h>0$, so $f(h)=\dfrac{6\cos h}{1+\sin h + \cos h}$ ~~and \begin{align*}
\ds\lim_{h \rightarrow 0^+} \dfrac{f(h)-f(0)}{h} &= \left.\ds\diff{}{x}\left\{\dfrac{6\cos x}{2+\sin x + \cos x}\right\}\right|_{x=0}
\\&=\left.\frac{-6\sin x(2+\sin x+\cos x)-6\cos x(\cos x-\sin x)}{(2+\sin x+\cos x)^2}\right|_{x=0}.\end{align*}

Since the limits from the left and right must be equal, this forces
$$
a=\frac{-6\sin 0(2+\sin 0+\cos 0)-6\cos 0(\cos 0-\sin 0)}{(2+\sin 0+\cos 0)^2}
=\frac{-6}{(2+1)^2}\implies\boxed{a=-\frac{2}{3}}
$$
\end{solution}


\begin{question}[2015Q]
For which values of $x$ does the derivative of $f(x) = \tan x$ exist?
\end{question}
\begin{hint}
There are infinitely many places where it does \emph{not} exist.
\end{hint}
\begin{answer}
All values of $x$ except $x=\frac{\pi}{2}+n\pi$, for any integer $n$.
\end{answer}
\begin{solution}
In order for $f'(x)$ to exist, $f(x)$ has to exist. We already know that $\tan x $ does not exist whenever $x=\frac{\pi}{2}+n\pi$ for any integer $n$. If we look a little deeper, since $\tan x = \frac{\sin x}{\cos x}$, the points where tangent does not exist correspond exactly to the points where cosine doesn't exist.

From its graph, tangent looks like a smooth curve over its domain, so we might guess that everywhere tangent is defined, its derivative is defined. We can check this: $f'(x) = \sec^2 x = \left(\frac{1}{\cos x}\right)^2$. Indeed, wherever $\cos x$ is nonzero, $f'$ exists.

So, $f'(x)$ exists for all values of $x$ \emph{except} when $x=\frac{\pi}{2}+n\pi$ for some integer $n$.
\end{solution}


\begin{Mquestion}[2015Q]
For what values of $x$ does the derivative of
$\dfrac{10\sin(x)}{x^2+x-6}$ exist? Explain your answer.
\end{Mquestion}
\begin{answer}
The function is differentiable whenever $x^2+x-6\ne 0$ since the derivative equals
\begin{align*}
\frac{10\cos(x)\cdot (x^2+x-6)-10\sin(x)\cdot (2x+1)}{(x^2+x-6)^2},
\end{align*}
which is well-defined unless $x^2+x-6=0$. We solve $x^2+x-6=(x-2)(x+3)=0,$
and get $x=2$ and $x=-3$. So, the function is differentiable for all real values $x$ except for $x=2$ and for $x=-3$.
\end{answer}
\begin{solution}
The function is differentiable whenever $x^2+x-6\ne 0$ since the derivative equals
\begin{align*}
\frac{10\cos(x)\cdot (x^2+x-6)-10\sin(x)\cdot (2x+1)}{(x^2+x-6)^2},
\end{align*}
which is well-defined unless $x^2+x-6=0$. We solve $x^2+x-6=(x-2)(x+3)=0,$
and get $x=2$ and $x=-3$. So, the function is differentiable for all real values $x$ except for $x=2$ and for $x=-3$.
\end{solution}



\begin{question}[test]
For what values of $x$ does the derivative of
$\dfrac{x^2+6x+5}{\sin(x)}$ exist? Explain your answer.
\end{question}
\begin{answer}
The function is differentiable whenever $\sin(x)\ne 0$ since the derivative equals
\begin{align*}
\frac{\sin(x)\cdot (2x+6) - \cos(x)\cdot (x^2+6x+5)}{(\sin x)^2},
\end{align*}
which is well-defined unless $\sin x = 0$. This happens when $x$ is an integer multiple
of $\pi$. So, the function is differentiable for all real values $x$ except $x=n\pi,$, where $n$ is any integer.
\end{answer}
\begin{solution}
The function is differentiable whenever $\sin(x)\ne 0$ since the derivative equals
\begin{align*}
\frac{\sin(x)\cdot (2x+6) - \cos(x)\cdot (x^2+6x+5)}{(\sin x)^2},
\end{align*}
which is well-defined unless $\sin x = 0$. This happens when $x$ is an integer multiple
of $\pi$. So, the function is differentiable for all real values $x$ except $x=n\pi,$, where $n$ is any integer.
\end{solution}


\begin{question}[2015Q]
 Find the equation of the line tangent to the graph of $y=\tan(x)$ at
$x=\dfrac{\pi}{4}$.
\end{question}
\begin{answer}
$y -  1 = 2\cdot (x-\pi/4)$
\end{answer}
\begin{solution}
We compute the derivative of $\tan(x)$ as being $\sec^2(x)$, which evaluated at
$x=\frac{\pi}{4}$ yields $2$. Since we also compute
$\tan(\pi/4)=1$, then the equation of the tangent line is
\begin{align*}
y -  1 = 2\cdot (x-\pi/4).
\end{align*}
\end{solution}


\begin{question}[2015Q]
Find the equation of the line tangent to the graph of $y=\sin(x)+\cos(x)+e^x$
at
$x=0$.
\end{question}
\begin{answer} $y=2x+2$
\end{answer}
\begin{solution}
We compute the derivative $y' = \cos(x)-\sin(x)+e^x$, which evaluated at
$x=0$ yields $1-0+1 = 2$. Since we also compute $y(0)=0+1+1=2$, the equation of the
tangent line is
\begin{align*}
y -  2 = 2(x-0)
\end{align*}
ie $y=2x+2$.
\end{solution}


\begin{question}
For which values of $x$ does the function $f(x)=e^x\sin x$ have a horizontal tangent line?
\end{question}
\begin{answer}
$x = \frac{3\pi}{4}+n\pi$ for any integer $n$.
\end{answer}
\begin{solution}
We are asked to solve $f'(x)=0$. That is, $e^x[\sin x +  \cos x]=0$. Since $e^x$ is always positive, that means we need to find all points where $\sin x + \cos x =0$. That is, we need to find all values of $x$ where $\sin x = - \cos x$. Looking at the unit circle, we see this happens whenever $x = \frac{3\pi}{4}+n\pi$ for any integer $n$.
\begin{center}
\begin{tikzpicture}
\YEaxis{2.75}{2.75}
\draw[thick] node[shape=circle, minimum size=4cm, draw, inner sep=0]{};
\draw[dashed] (2.75,-2.75)--(-2.75,2.75);
\draw (1.41,-1.41) node[vertex, label=right:{$\sin (-\pi/4) = -\cos (-\pi/4)$}]{};
\draw (-1.41,1.41) node[vertex, label=left:{$\sin (3\pi/4) = -\cos (3\pi/4)$}]{};
\end{tikzpicture}
\end{center}
\end{solution}

%%%
%%% Andrew - extra problem starts here
%%%

\begin{question}
Let
\[f(x)=\left\{\begin{array}{ccc}
\frac{\sin x}{x}&,&x \neq 0\\
1&,&x=0
\end{array}\right.\]
Find $f'(0)$, or show that it does not exist.
\end{question}
\begin{hint}
You can set up the derivative using the limit definition: $f'(0)=\ds\lim_{h \to 0}\dfrac{f(h)-f(0)}{h}$. If the limit exists, it gives you $f'(0)$; if the limit does not exist, you conclude $f'(0)$ does not exist.

To evaluate the limit, recall that when we differentiated sine, we learned that for $h$ near 0,
\[\cos h \leq \frac{\sin h}{h}\leq 1\]
\end{hint}
\begin{answer}
$f'(0)=0$
\end{answer}
\begin{solution}
First, we note that our function is continuous, because
\[\lim_{x \to 0} f(x)=\lim_{x \to 0} \frac{\sin x}{x}=1=f(0)\]
This is a handy thing to check: if the function were discontinuous at $x=0$, then we would automatically know that it was not differentiable there.

Now, on to the derivative. We can use the limit definition:
\begin{align*}
f'(0)&=\lim_{h \to 0}\frac{f(0+h)-f(0)}{h}\qquad\mbox{ if it exists}\\
&=\lim_{h \to 0} \frac{f(h)-1}{h}\\
&=\lim_{h \to 0} \frac{\frac{\sin h}{h}-1}{h}
\end{align*}
As $h$ approaches 0, both the numerator and the denominator approach 0. So, to evaluate the limit, we need to do more work. The key insight we can use is a result that was shown in the text while evaluating the derivative of sine. When $h$ is close to 0, $\cos h \leq \frac{\sin h}{h} \leq 1$. We use this to bound our limit, and then apply the squeeze theorem.
\begin{align*}
&\cos h&&\leq&&\frac{\sin h}{h}&&\leq&&1\\
\mbox{So, }\qquad&\frac{\cos h-1}{h}&&\leq&&\frac{\frac{\sin h}{h}-1}{h}&&\leq&&\frac{1-1}{h}=0
\end{align*}
We evaluate the limits of the bounds. For the lower bound, notice its similarity to the definition of the derivative.
\begin{align*}
\lim_{h \to 0}\frac{\cos h-1}{h}&=\lim_{h \to 0}\frac{\cos(0+h)-\cos(0)}{h}\\
&=\diff{}{x}\left.\left\{\cos x \right\}\right|_{x=0}=0\\
\lim_{h \to 0}0&=0
\end{align*}
So, by the Squeeze Theorem, $f'(0)=\ds\lim_{h \to 0}\frac{\frac{\sin h}{h}-1}{h}=0$.
\end{solution}

%%%
%%% Andrew - extra problem ends here
%%%


%%%%%%%%%%%%%%%%%%
\subsection*{Application of Understanding}
%%%%%%%%%%%%%%%%%%


\begin{question}[2010H]
Differentiate the function \[h(x) = \sin(|x|)\] and give the domain where the derivative exists.
\end{question}
\begin{hint}
Recall $|x|=\left\{\begin{array}{rl}
x&x\ge 0\\
-x&x<0
\end{array}\right.$. To determine whether $h(x)$ is differentiable at $x=0$, use the definition of the derivative.
\end{hint}
\begin{answer}
$h'(x)=\left\{\begin{array}{rl}
\cos x&x> 0\\
-\cos x&x<0
\end{array}\right.$ It exists for all $x \neq 0$.
\end{answer}
\begin{solution}
As usual, when dealing with the absolute value function, we can make things a little clearer by splitting it up into two pieces.
\begin{align*}
|x|&=\left\{\begin{array}{rl}
x&x\ge 0\\
-x&x<0
\end{array}\right.\intertext{So,}
\sin|x|&=\left\{\begin{array}{rl}
\sin x&x\ge 0\\
\sin(-x)&x<0
\end{array}\right.
=\left\{\begin{array}{rl}
\sin x&x\ge 0\\
-\sin x&x<0
\end{array}\right.
\intertext{where we used the identity $\sin(-x)=-\sin x$. From here, it's easy to see $h'(x)$ when $x$ is anything \emph{other than} zero.}
\diff{}{x}\{\sin|x|\}&=\left\{\begin{array}{rl}
\cos x&x> 0\\
??&x=0\\
-\cos x&x<0
\end{array}\right.
\intertext{To decide whether $h(x)$ is differentiable at $x=0$, we use the definition of the derivative. One word of explanation: usually in the definition of the derivative, $h$ is the tiny ``change in $x$" that is going to zero. Since $h$ is the name of our function, we need another letter to stand for the tiny change in $x$, the size of which is tending to zero. We chose $t$.}
\lim_{t \to 0} \frac{h(t+0)-h(0)}{t}&=\lim_{t \to 0}\frac{\sin|t|}{t}
\intertext{We consider the behaviour of this function to the left and right of $t=0$:}
\frac{\sin |t|}{t}&=\left\{\begin{array}{ll}
\frac{\sin t}{t} & t\ge 0\\
\frac{\sin (-t)}{t} & t <0
\end{array}\right.
=\left\{\begin{array}{ll}
\frac{\sin t}{t} & t\ge 0\\
-\frac{\sin t}{t} & t <0
\end{array}\right.
\intertext{Since we're evaluating the limit as $t$ goes to zero, we need the fact that $\ds\lim_{t \to 0}\dfrac{\sin t}{t}=1$.
We saw this in Section~\ref*{sec exp func}, but also we know enough now to evaluate it another way. Using the definition of the derivative: }
\lim_{t \to 0}\frac{\sin t}{t}&=\lim_{t \to 0}\frac{\sin (t+0)-\sin (0)}{t}=\left.\diff{}{x}\{\sin x\}\right|_{t=0}=\cos 0=1
\intertext{At any rate, since we know $\ds\lim_{t \to 0}\dfrac{\sin t}{t}=1$, then:}
\lim_{t \to 0^+} \frac{h(t+0)-h(0)}{t}&=\lim_{t \to 0^+}\frac{\sin t}{t}=1\qquad
\lim_{t \to 0^-} \frac{h(t+0)-h(0)}{t}=\lim_{t \to 0^-}\frac{-\sin t}{t}=-1
\intertext{So, since the one-sided limits disagree,}
\lim_{t \to 0} \frac{h(t+0)-h(0)}{t}&=DNE
\intertext{so $h(x)$ is not differentiable at $x=0$. Therefore,}
h'(x)&=\left\{\begin{array}{rl}
\cos x&x> 0\\
-\cos x&x<0
\end{array}\right.
\end{align*}

\end{solution}


\begin{Mquestion}[2006H]
For the function
\[
f(x) =\left\{\begin{array}{ll} 0 & x\le 0\\
               \frac{\sin(x)}{\sqrt{x}} & x > 0\end{array}\right.
\]

which of the following statements is correct?
\begin{enumerate}[i.]
\item\label{s2.8_2006i} $f$ is undefined at $x = 0$.
\item\label{s2.8_2006ii}  $f$ is neither continuous nor differentiable at $x = 0$.
\item\label{s2.8_2006iii}  $f$ is continuous but not differentiable at $x = 0$.
\item\label{s2.8_2006iv} $f$ is differentiable but not continuous at $x = 0$.
\item\label{s2.8_2006v} $f$ is both continuous and differentiable at $x = 0$.
\end{enumerate}
\end{Mquestion}
\begin{hint} To decide whether the function is differentiable, use the definition of the derivative.
\end{hint}
\begin{answer} \ref{s2.8_2006iii}
\end{answer}
\begin{solution}
Statement \ref{s2.8_2006i} is false, since $f(0)=0$.
Statement \ref{s2.8_2006iv} cannot hold, since a function that is differentiable is also continuous.

Since $\ds\lim_{x\rightarrow 0+}\frac{\sin x}{x}=1$ (we saw this in Section~\ref*{sec diff trig}%2.8
),
\begin{align*}
\lim_{x\rightarrow 0+}f(x)&=\lim_{x\rightarrow 0+}\frac{\sin x}{\sqrt{x}}\\
&=\lim_{x\rightarrow 0+}\sqrt{x}\frac{\sin x}{x}\\&=0\cdot 1=0
\intertext{So $f$ is continuous at $x=0$, and so
Statement \ref{s2.8_2006ii} does not hold. Now, let's consider $f'(x)$.}
\lim_{x\rightarrow 0+}\frac{f(x)-f(0)}{x}
&=\lim_{x\rightarrow 0+}\frac{\frac{\sin x}{\sqrt{x}}-0}{x}\\
&=\lim_{x\rightarrow 0+}\frac{1}{\sqrt{x}}\frac{\sin x}{x}=+\infty
\intertext{Therefore, using the definition of the derivative,}
f'(0)&=\lim_{x \to 0}\frac{f(x)-f(0)}{x}~~\mbox{ if it exists, but}\\
\lim_{x \to 0}\frac{f(x)-f(0)}{x}&=DNE
\end{align*}
since one of the one-sided limits does not exist. So $f$ is continuous but not differentiable
at $x=0$. The correct statement is \ref{s2.8_2006iii}.
\end{solution}





\begin{question}[2011H]
Evaluate $\lim\limits_{x\rightarrow 0}
                 \dfrac{\sin x^{27}+2x^5 e^{x^{99}}}{\sin^5 x}$.
\end{question}
\begin{hint}
In this chapter, we learned $\ds\lim_{x \to 0}\dfrac{\sin x}{x}=1$. If you divide the numerator and denominator by $x^5$, you can make use of this knowledge.
\end{hint}
\begin{answer} $2$
\end{answer}
\begin{solution}
Recall that $\lim\limits_{x\rightarrow 0}\dfrac{\sin x}{x} =1$. In order to take advantage of this knowledge, we divide the numerator and denominator by $x^5$ (because $5$ is the power of sine in the denominator, and a denominator that goes to zero generally makes a limit harder).
\begin{align*}
\lim_{x\rightarrow 0}  \dfrac{\sin x^{27}+2x^5 e^{x^{99}}}{\sin^5 x}
&=\lim_{x\rightarrow 0}  \dfrac{\dfrac{\sin x^{27}}{x^{5}}+2 e^{x^{99}}}
                                {\left(\dfrac{\sin x}{x}\right)^5}
\end{align*}
Now the denominator goes to 1, which is nice, but we need to take care of the fraction $\dfrac{\sin x^{27}}{x^5}$ in the numerator. This fraction isn't very familiar, but we know that, as $x$ goes to zero, $x^{27}$ also goes to zero,
so that $\dfrac{\sin x^{27}}{x^{27}}$ goes to $1$. Consequently,
\begin{align*}
\lim_{x\rightarrow 0}  \dfrac{\sin x^{27}+2x^5 e^{x^{99}}}{\sin^5 x}
&=\lim_{x\rightarrow 0}  \dfrac{x^{22}\dfrac{\sin x^{27}}{x^{27}}+2 e^{x^{99}}}
                                {\left(\dfrac{\sin x}{x}\right)^5}
=\dfrac{0\times 1+2\times e^0}{1^5}
=2
\end{align*}

\end{solution}
