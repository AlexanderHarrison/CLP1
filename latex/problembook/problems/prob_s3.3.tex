%
% Copyright 2018 Joel Feldman, Andrew Rechnitzer and Elyse Yeager.
% This work is licensed under a Creative Commons Attribution-NonCommercial-ShareAlike 4.0 International License.
% https://creativecommons.org/licenses/by-nc-sa/4.0/
%
\questionheader{ex:s3.3}


%%%%%%%%%%%%%%%%%%
\subsection*{More Section 3.3 Problems}
%%%%%%%%%%%%%%%%%%

\begin{question}[2009H]
Find $f(2)$ if $f'(x) = \pi f(x)$ for all $x$, and $f(0) = 2$.
\end{question}
\begin{hint}
Use Theorem~\ref*{thm:growthDEsoln} to figure out what $f(x)$ looks like.
\end{hint}
\begin{answer} $f(2)=2e^{2\pi}$
\end{answer}
\begin{solution}
The first piece of information given tells us $\ds\diff{f}{x}=\pi f(x)$. Then by
Theorem~\ref*{thm:growthDEsoln},
\begin{align*}
f(x)&=Ce^{\pi x}
\intertext{for some constant $C$. The second piece of given information tells us
$f(0)=2$. Using this, we can solve for $C$:}
2&=f(0)=Ce^{0}=C
\intertext{Now, we know $f(x)$ entirely:}
f(x)&=2e^{\pi t}
\intertext{So, we can evaluate $f(2)$}
f(2)&=2e^{2\pi}
\end{align*}
\end{solution}


\begin{question}
Which functions $T(t)$ satisfy the differential equation
$\ds\diff{T}{t}=7T+9$?
\end{question}
\begin{hint}
To use  Corollary~\ref*{cor:coolingDEsoln}, you need to re-write the differential equation as
\[\diff{T}{t}=7\left[T-\left(-\frac{9}{7}\right)\right].\]
\end{hint}
\begin{answer}
Solutions to the differential equation have the form
\[T(t)=\left[T(0)+\frac{9}{7}\right]e^{7t}-\frac{9}{7}\]
for some constant $T(0)$.
\end{answer}
\begin{solution}
To use  Corollary~\ref*{cor:coolingDEsoln}, we re-write the differential equation as
\[\diff{T}{t}=7\left[T-\left(-\frac{9}{7}\right)\right].\]
Now, $A=-\dfrac{9}{7}$ and $K=7$, so we see that the solutions to the differential equation have the form
\[T(t)=\left[T(0)+\frac{9}{7}\right]e^{7t}-\frac{9}{7}\]
for some constant $T(0)$.
\begin{align*}
\intertext{We can check that this is reasonable: if }
T(t)&=\left[T(0)+\frac{9}{7}\right]e^{7t}-\frac{9}{7}
\intertext{then}
\diff{T}{t}&=7\left[T(0)+\frac{9}{7}\right]e^{7t}\\
&=7\left[T+\frac{9}{7}\right]\\
&=7T+9.
\end{align*}
\end{solution}



\begin{question}[1998H]
It takes 8 days for 20\% of a particular
radioactive material to decay. How long does it take for 100 grams of the
material to decay to 40 grams?
\end{question}
\begin{hint}
The amount of the material at time $t$ will be $Q(t)=Ce^{-kt}$\\ for some constants $C$ and $k$.
\end{hint}
\begin{answer}
$\dfrac{8\log(0.4)}{\log(0.8)}\approx 32.85$ days
\end{answer}
\begin{solution}
Let $Q(t)$ denote the amount of radioactive material after $t$ days.
Then
$Q(t)=Q(0)e^{kt}$. We are  told
\begin{align*}
Q(8)&=0.8\,Q(0)
\intertext{So,}
Q(0)e^{8k}&=0.8\,Q(0)\\
e^{8k}&=0.8\\
e^k&=0.8^{\tfrac{1}{8}}
\intertext{If $Q(0)=100$, the time $t$ at which $Q(t)=40$ is determined by}
40=Q(t)&=Q(0)e^{kt}=100e^{kt}=100\left(0.8^{\tfrac{1}{8}}\right)^t=100\cdot 0.8^{\tfrac{t}{8}}
\intertext{Solving for $t$:}
\frac{40}{100}&=0.8^{\tfrac{t}{8}}\\
\log\left(0.4\right)&=\log\left(0.8^{\tfrac{t}{8}}\right)=\frac{t}{8}\log(0.8)\\
t&=\frac{8\log(0.4)}{\log(0.8)}\approx 32.85\mbox{ days}
\end{align*}
100 grams will decay to 40 grams in about 32.85 days.
\end{solution}


\begin{question}
A glass of boiling water is left in a room. After 15 minutes, it has cooled to 85$^\circ$ C, and after 30 minutes it is 73$^\circ$ C. What temperature is the room?
\end{question}
\begin{hint}
In your calculations, it might come in handy that $e^{30K}=\left(e^{15K}\right)^2$.
\end{hint}
\begin{answer}
25$^\circ$ C
\end{answer}
\begin{solution}
Let $t=0$ be the time the boiling water is left in the room, and let $T(t)$ be the temperature of the water $t$ minutes later, so $T(0)=100$.
Using Newton's Law of Cooling, we model the temperature of the water at time $t$ as
\begin{align*}
T(t)&=[100-A]e^{Kt}+A
\intertext{where $A$ is the room temperature, and $K$ is some constant. We are told that $T(15)=85$ and $T(30)=73$, so:}
85=T(15)&=[100-A]e^{15K}+A\\
73=T(30)&=[100-A]e^{30K}+A
\intertext{Rearranging both equations:}
\frac{85-A}{100-A}&=e^{15K}\\
\frac{73-A}{100-A}&=e^{30K}=\left(e^{15K}\right)^2
\intertext{Using these equations:}
\left(\frac{85-A}{100-A}\right)^2=\left(e^{15K}\right)^2&=e^{30K}=\frac{73-A}{100-A}\\
\frac{(85-A)^2}{100-A}&=73-A\\
(85-A)^2&=(73-A)(100-A)\\
85^2-170A+A^2&=7300-173A+A^2\\
173A-170A&=7300-85^2\\
3A&=75\\
A&=25
\end{align*}
The room temperature is 25$^\circ$ C.
\end{solution}


\begin{Mquestion}[1997D]
  A 25-year-old graduate of UBC is given \$50,000 which is invested
at 5\% per year compounded continuously. The graduate also intends to
deposit money continuously at the rate of \$2000 per year. Assuming that
the interest rate remains 5\%, the amount $A(t)$ of money at time $t$ satisfies
the equation
$$
\diff{A}{t}= 0.05 A+2000
$$
\begin{enumerate}[(a)]
\item Solve this equation and determine the amount of money in the account when the graduate
is 65.
\item At age 65, the graduate will withdraw money continuously
at the rate of $W$ dollars per year. If the money must last until the person
is 85, what is the largest possible value of $W$?
\end{enumerate}
\end{Mquestion}
\begin{hint}
The differential equation in the problem has the same form as the differential equation from Newton's Law of Cooling.
\end{hint}
\begin{answer}
\begin{enumerate}[(a)]
\item $A(t)=90,\!000\cdot e^{0.05t}-40,\!000$ \\
When the graduate is 65, they will have \$625,015.05 in the account.
\item \$49,437.96
\end{enumerate}
\end{answer}
\begin{solution}
\begin{enumerate}[(a)]
\item The amount of money at time $t$ obeys
\begin{align*}
\diff{A}{t}&= 0.05 A+2,\!000=0.05[A-(-40,\!000)]
\intertext{
Using Corollary~\ref*{cor:coolingDEsoln},}
A(t)&=[A(0)+40,\!000]e^{0.05t}-40,\!000\\
&=90,\!000\cdot e^{0.05t}-40,\!000
\intertext{where $t=0$ corresponds to the year when the graduate is 25.}
\intertext{When the graduate is 65 years old, $t=40$, so}
A(40)&=90,\!000\, e^{0.05 \times 40}-40,000\approx \$625,015.05
\end{align*}
\item
When the graduate stops depositing money and instead
starts withdrawing money at a rate $W$, the equation for $A$ becomes
\begin{align*}
\diff{A}{t}&= 0.05 A-W= 0.05 [A(t)-20 W]
\intertext{
Using Corollary~\ref*{cor:coolingDEsoln}, and
assuming that the interest rate remains 5\%,}
A(t)&=[A(0)-20W]e^{0.05t}+20W\\
&=[625,015.05-20W]e^{0.05t}+20W
\intertext{Note that, for part (b), we only care about what happens when the graduate starts withdrawing money. We take $t=0$ to correspond to the year when the graduate is 65--so we're using a different $t$ from part (a). Then from part (a), $A(0)=625,025.05$.}
\intertext{We want the account to be depleted when the graduate is 85. So, we
want }
0&=A(20)\\
0&=20W+ e^{0.05\times 20}(625,015.05-20W)
\\
0&=20W+ e(625,015.05-20W)\\
20(e-1)W&= 625,015.05e\\
W&=\frac{625,015.05e}{20(e-1)}\approx\$49,437.96
\end{align*}
\end{enumerate}
\end{solution}


\begin{question}[1996D]
  An investor puts \$120,000 which into a bank account which pays
6\% annual interest, compounded continuously. She plans to withdraw
money continuously from the account at the rate of \$9000 per year. If
 $A(t)$ is the amount of money at time $t$, then
$$
\diff{A}{t}= 0.06 A-9000
$$
\begin{enumerate}[(a)]
\item\label{s3.3investor1} Solve this equation for $A(t)$.
\item\label{s3.3investor2} When will the money run out?
\end{enumerate}
\end{question}
\begin{hint} We know the form of the solution $A(t)$ from Corollary~\ref*{cor:coolingDEsoln}.
\end{hint}
\begin{answer}
\eqref{s3.3investor1} $A(t)=150,\!000-30,\!000\, e^{0.06 t}$
\hspace{1cm}\eqref{s3.3investor2} after {26.8 yrs}
\end{answer}
\begin{solution}
\eqref{s3.3investor1}
The amount of money at time $t$ obeys
\begin{align*}
\diff{A}{t}&= 0.06 A-9,\!000=0.06[A-150,\!000]
\intertext{Using Corollary~\ref*{cor:coolingDEsoln},}
A(t)&=[A(0)-150,\!000]e^{0.06t}+150,\!000\\
&=[120,\!000-150,\!000]e^{0.06t}+150,\!000\\
&=-30,\!000e^{0.06t}+150,\!000
\end{align*}

\eqref{s3.3investor2}
The money runs out when $A(t)=0$.
\begin{align*}
A(t)&=0\\
150,\!000-30,\!000\, e^{0.06 t}&=0\\
30,\!000\, e^{0.06t}&=150,\!000\\
e^{0.06 t}&=5\\
0.06 t&=\log 5\\
t&=\frac{\log 5}{0.06}\approx\mbox{26.8 yrs}
\end{align*}
The money runs out in about 26.8 years.

Remark: without earning any interest, the money would have run out in about 13.3 years.
\end{solution}


\begin{question}[1997D]A particular bacterial culture grows at a rate proportional
to the number of bacteria present. If the size of the culture triples every
nine hours, how long does it take the culture to double?
\end{question}
\begin{hint}
If a function's rate of change is proportional to the function itself, what does the function looks like?
\end{hint}
\begin{answer} $\dfrac{9\log 2}{\log 3}\approx5.68$ hr
\end{answer}
\begin{solution}
Let $Q(t)$ denote the number of bacteria at time $t$. We are told that
$Q'(t)=k Q(t)$ for some constant of proportionality $k$. Consequently,
$Q(t)=Q(0)e^{kt}$ (Corollary~\ref*{thm:growthDEsoln}). We are also told
\begin{align*}
Q(9)&=3Q(0) \\
\mbox{So,}\qquad Q(0)e^{9k}&=3Q(0)\\
e^{9k}&=3\\
e^{k}&=3^{\tfrac{1}{9}}
\intertext{The doubling time $t$ obeys:}
Q(t)&=2Q(0)\\
\mbox{So,}\qquad Q(0)e^{kt}&=2Q(0)\\
e^{kt}&=2\\
3^{\tfrac{t}{9}}&=2\\
\frac{t}{9}\log 3 &=\log 2\\
t&=9\frac{\log 2}{\log 3}\approx5.68 \mbox{ hr}
\end{align*}
\end{solution}



\begin{question}[2012H]
An object falls under gravity near the surface of the earth
and its motion is impeded by air resistance proportional to its speed. Its
velocity $v$ satisfies the differential equation
$$
\dfrac{dv}{dt}=-g-kv
$$
where $g$ and $k$ are positive constants.
\begin{enumerate}[(a)]
\item Find the velocity of the object as a function of time $t$,
given that it was $v_0$ at $t=0$.
\item Find $\lim\limits_{t\rightarrow\infty} v(t)$.
\end{enumerate}
\end{question}
\begin{hint}
The equation from Newton's Law of Cooling, in Corollary~\ref*{cor:coolingDEsoln},
has a similar form to the differential equation in this question.
\end{hint}
\begin{answer}
(a) $v(t)=\left[v_0+\frac{g}{k}\right]e^{-kt}-\frac{g}{k}$\\
(b) $\lim\limits_{t\rightarrow\infty} v(t)=-\dfrac{g}{k}$
\end{answer}
\begin{solution}
(a)  We want our differential equation to have the format of the equation in
Corollary~\ref*{cor:coolingDEsoln}:
\begin{align*}
\frac{dv}{dt}(t)&=-g-kv(t)\\
&=-k\left(v(t)+\frac{g}{k}\right)\\
&=-k\left(v(t)-\left(-\frac{g}{k}\right)\right)
\intertext{So, we can use the corollary, with $K=-k$, $T=v$, and $A=-\dfrac{g}{k}$.}
v(t)&=\left(v(0)-\left(-\frac{g}{k}\right)\right)e^{-kt}-\frac{g}{k}\\
&=\left(v_0+\frac{g}{k}\right)e^{-kt}-\frac{g}{k}
\end{align*}


(b)
\begin{align*}
\lim\limits_{t\rightarrow\infty} v(t)
&=\lim_{t \to \infty}\left[\left(v_0+\frac{g}{k}\right)e^{-kt}-\frac{g}{k}\right]\\
&=\left(v_0+\frac{g}{k}\right)\left(\lim_{t \to \infty}e^{-kt}\right)-\frac{g}{k}
\intertext{Since $k$ is \emph{positive}:}
&=\left(v_0+\frac{g}{k}\right)\left(0\right)-\frac{g}{k}
\\
&=-\dfrac{g}{k}\end{align*}

Remark: This means, as the object falls, instead of accelerating without bound, it approaches some maximum speed. The velocity is negative because the object is moving in the negative direction--downwards.
\end{solution}
