%
% Copyright 2018 Joel Feldman, Andrew Rechnitzer and Elyse Yeager.
% This work is licensed under a Creative Commons Attribution-NonCommercial-ShareAlike 4.0 International License.
% https://creativecommons.org/licenses/by-nc-sa/4.0/
%
\questionheader{ex:s1.1}

%%%%%%%%%%%%%%%%%%
\subsection*{\Conceptual}
%%%%%%%%%%%%%%%%%%
\begin{Mquestion}
On the graph below, draw:
\begin{enumerate}[(a)]
\item\label{s1.1p1.1} The tangent line to $y=f(x)$ at $P$,
\item\label{s1.1p1.2}  the tangent line to $y=f(x)$ at $Q$, and
\item\label{s1.1p1.3}  the secant line to $y=f(x)$ through $P$ and $Q$.
\end{enumerate}
\begin{center}
\begin{tikzpicture}
\YEaaxis{1}{5}{1}{3}
\draw[thick] plot[domain=-1:5, samples=100](\x,{(3*\x-10)*(\x+1)*(\x-2)/30}) node[right]{$y=f(x)$};
\draw (1,7/15) node[vertex, label=above:$P$]{};
\draw (4.5,7*11/48) node[vertex, label= left:$Q$]{};
\end{tikzpicture}
\end{center}
\end{Mquestion}
\begin{answer}
\begin{center}
\begin{tikzpicture}
\YEaaxis{1}{5}{1}{3}
\draw[thick] plot[domain=-1:5, samples=100](\x,{(3*\x-10)*(\x+1)*(\x-2)/30}) node[right]{$y=f(x)$};
\draw (1,7/15) node[vertex, label=above:$P$]{};
\draw (4.5,7*11/48) node[vertex, label= left:$Q$]{};
\draw[thick, blue] plot[domain=-1:3.5] (\x,{27/30-13/30*\x}) node[below left]{\ref{s1.1p1.1}};
\draw[thick, green] plot[domain=3.5:5.5] (\x,{7*11/48+2.31*(\x-4.5)}) node[left]{\ref{s1.1p1.2}};
\draw[thick, red] plot[domain=0:5.5] (\x,{7/15+.325*(\x-1)}) node[right]{\ref{s1.1p1.3}};
\end{tikzpicture}
\end{center}
\end{answer}
\begin{solution}
\begin{center}
\begin{tikzpicture}
\YEaaxis{1}{5}{1}{3}
\draw[thick] plot[domain=-1:5, samples=100](\x,{(3*\x-10)*(\x+1)*(\x-2)/30}) node[right]{$y=f(x)$};
\draw (1,7/15) node[vertex, label=above:$P$]{};
\draw (4.5,7*11/48) node[vertex, label= left:$Q$]{};
\draw[thick, blue] plot[domain=-1:3.5] (\x,{27/30-13/30*\x}) node[below left]{\ref{s1.1p1.1}};
\draw[thick, green] plot[domain=3.5:5.5] (\x,{7*11/48+2.31*(\x-4.5)}) node[left]{\ref{s1.1p1.2}};
\draw[thick, red] plot[domain=0:5.5] (\x,{7/15+.325*(\x-1)}) node[right]{\ref{s1.1p1.3}};
\end{tikzpicture}
\end{center}
The tangent line to $y=f(x)$ at a point should go through
  the point, and be ``in the same direction" as $f$ at that point. The secant line through $P$ and $Q$ is simply the straight line passing through $P$ and $Q$.
\end{solution}

%
\begin{Mquestion}
Suppose a curve $y=f(x)$ has tangent line $y=2x+3$ at the point $x=2$.
\begin{enumerate}[(a)]
\item True or False: $f(2)=7$
\item True or False: $f(3)=9$
\end{enumerate}
\end{Mquestion}
\begin{hint}
The tangent line to a curve at point $P$ passes through $P$.
\end{hint}
\begin{answer}
\begin{enumerate}[(a)]
\item True
\item In general, this is false.
For most functions $f(x)$ this will be false, but there
                are some functions for which it is true.
\end{enumerate}
\end{answer}
\begin{solution}
\begin{enumerate}[(a)]
\item\label{test} True: since $y=2x+3$ is the tangent line to $y=f(x)$ \emph{at the point} $x=2$, this means the function and the tangent line have the same value at $x=2$. So $f(2)=2(2)+3=7$.
\item In general, this is false. We are only guaranteed
                that the curve $y=f(x)$ and its tangent line $y=2x+3$
                agree at $x=2$. The functions $f(x)$ and $2x+3$ may or
                may not take the same values when $x\ne 2$. For example,
                if $f(x)=2x+3$, then of course $f(x)$ and $2x+3$ agree
                for all values of $x$. But if $f(x) = 2x+3 +(x-2)^2$,
                then $f(x)$ and $2x+3$ agree only for $x=2$.
\end{enumerate}
\end{solution}


%
\begin{question}
Let $L$ be the tangent line to a curve $y=f(x)$ at some point $P$. How many times will $L$ intersect the curve $y=f(x)$?
\end{question}
\begin{hint}
Try drawing tangent lines to the following curves, at the given points $P$:
\begin{center}
\begin{tikzpicture}[scale=0.8]
\YEaaxis{2}{2}{1}{2}
\draw[thick] plot[domain=-1.5:1.5](\x,\x*\x) node[right]{$y=f(x)$};
\draw (0,0) node[vertex, label=below right:$P$]{};
\end{tikzpicture}
\hfill
\begin{tikzpicture}[scale=0.8]
\YEaaxis{2}{2}{1}{2}
\draw[thick] plot[smooth] coordinates {(-2,-1) (-1,.5)  (1,-.5) (2,2)} node[right]{$y=f(x)$};
\draw (-1,.5) node[vertex, label=above right:$P$]{};
\end{tikzpicture}
\hfill
\begin{tikzpicture}[scale=0.8]
\YEaaxis{2}{2}{2}{2}
\draw[thick] plot[domain=-2:1.75, samples=200] (\x, {(3/8*\x+5/4)*sin(\x*10 r)}) node[right]{$y=f(x)$};
\draw (-1.73,.6) node[vertex, label=above:$P$]{};
\end{tikzpicture}
\end{center}
\end{hint}
\begin{answer}
At least once.
\end{answer}
\begin{solution}
Since the tangent line to the curve at point $P$ passes through point $P$, the curve and the tangent line touch at point $P$. So, they must intersect at least once. By drawing various examples, we can see that different curves may touch their tangent lines exactly once, exactly twice, exactly three times, etc.
\begin{center}
\begin{tikzpicture}
\YEaaxis{2}{2}{1}{2}
\draw[thick] plot[domain=-1.5:1.5](\x,\x*\x) node[right]{$y=f(x)$};
\draw (0,0) node[vertex, label=below right:$P$]{};
\draw[blue, very thick] (-1.75,0)--(1.75,0);
\draw (0,-1.5) node{Blue tangent line touches curve only once, at $P$};
\end{tikzpicture}
\hfill
\begin{tikzpicture}
\YEaaxis{2}{2}{1}{2}
\draw[thick] plot[smooth] coordinates {(-2,-1) (-1,.5)  (1,-.5) (2,2)} node[right]{$y=f(x)$};
\draw (-1,.5) node[vertex, label=above right:$P$]{};
\draw[blue, very thick] (-2,.5)--(2,.5);
\draw (0,-1.5) node{Blue tangent line touches curve twice};
\end{tikzpicture}
\hfill
\begin{tikzpicture}
\YEaaxis{2}{2}{2}{2}
\draw[thick] plot[domain=-2:1.75, samples=200] (\x, {(3/8*\x+5/4)*sin(\x*10 r)}) node[right]{$y=f(x)$};
\draw (-1.73,.6) node[vertex, label=above:$P$]{};
\draw[blue, very thick] (-2,.6)--(2,.6);
\draw (0,-3.5) node{Blue tangent line touches curve many times};
\end{tikzpicture}
\end{center}
\end{solution}
